%% The following is a directive for TeXShop to indicate the main file
%%!TEX root = diss.tex

\chapter{Abstract}

Users desire control over their data even as they share them across device
  boundaries. At the moment, they rely on ad-hoc solutions such as sending
  self-destructible data with ephemeral messaging apps such as SnapChat. We 
  present \textbf{Trusted Capsules}, a general cross-device access
  control abstraction for files. It bundles sensitive files with the policies
  that govern their accesses into units we call {\em capsules}. Capsules appear
  as regular files in the system. When an app opens one, its policy is executed
  in ARM TrustZone, a hardware-based trusted execution environment, to determine
  if access should be allowed or denied. As Trusted Capsules is based on a
  pragmatic threat model, it works with unmodified apps that users have come to
  rely on, unlike existing work. We show that policies in Trusted Capsules are
  expressible and that the slowdowns in our approach are limited to the opening
  and closing of capsules. Once an app opens a capsule, its read throughput of
  the file is identical to regular non-capsule files.


% Consider placing version information if you circulate multiple drafts
%\vfill
%\begin{center}
%\begin{sf}
%\fbox{Revision: \today}
%\end{sf}
%\end{center}
