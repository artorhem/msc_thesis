%%%%%%%%%%%%%%%%%%%%%%%%%%%%%%%%%%%%%%%%%%%%%%%%%%%%%%%
\chapter{Use case examples}
\label{sec:new-policies}
%%%%%%%%%%%%%%%%%%%%%%%%%%%%%%%%%%%%%%%%%%%%%%%%%%%%%%%

In this section, we discuss several use cases to highlight the capabilities of
Trusted Capsules. 

% In this section, we introduce four policies that demonstrate how trusted
% capsules can be used in real-life scenarios that require the enforcement of
% strong first-use policies. We use these policies to evaluate the system in
% section \ref{sec:eval}, and to demonstrate the expressiveness of our policy API
% as introduced in Section \ref{subsec:policy}.


\textbf{Access control based on time or location}: Enterprises may wish to
restrict employees from opening company files outside the office or a user may
require his family members to view shared pictures only at their
homes. Alternatively, the data owner may wish to allow access to sensitive
content only within a pre-determined time period. Such requirements are
straightforward to express in our system. When a capsule policy's
\texttt{evaluate\_policy()} function is evaluated at the time of
\texttt{open()}, it can access the device location and time \footnote{Information such as device location can only be trusted if the device driver is located within TrustZone and the peripheral bus allows exclusive access to the sensors. There has been prior work that achieves this. The Trusted Capsules prototype does not do this, and relies on the Normal World OS to provide this information. We leave this as future work.} to decide if the
access should be allowed or denied. Alternatively, instead of simply denying access
to a capsule, policies may use the \texttt{redact()} API in
Table~\ref{Tbl:lua_ext} to allow access but with sensitive content
redacted.

For example, Figure~\ref{fig:location_policy} illustrates a policy
that denies access to the capsule if the location from which it is
being accessed is outside the specified location range. 

\textbf{Requiring permissions in real time}: In some cases, users may wish to
have real-time control over their data. For example, Alvin may wish to be asked
each time Barbara opens his capsule whether or not to allow her access. It is
straightforward to support this scenario in Trusted Capsules as policies can
communicate with remote servers over the Internet.

We implemented this scenario in our prototype using Twitter. When a user opens a capsule,
the policy uses the \texttt{getState()} API method to communicate with a custom
server and passes the Twitter handle of the capsule owner. The server then sends
a direct Twitter message to the owner of the capsule with an access request
and asks him to respond with a ``yes'' or ``no'' to approve or decline the access,
respectively. The server returns the owner's decision to the policy and the
appropriate action is taken. At the moment, the Twitter message to the owner
does not identify the user trying to open the capsule but this can be
implemented by mapping unique device identifiers to Twitter handles.

\textbf{Progressive trust}: The APIs in Table~\ref{Tbl:lua_ext} may be composed
to support other useful scenarios. Suppose Bob wants to share
sensitive data with someone but does not yet completely trust that person. He
can use a policy that contacts a remote server to log access attempts and to
identify what data should be returned to the app. Initially, Bob may choose to
provide a heavily redacted version of the data (e.g., an image with blurred-out
faces or a document with key sections removed). As his trust towards
the person grows, he can progressively share more sensitive content by recording his
decisions on the server.

% \textbf{Pre-distribution of content}: Popular TV shows such as Game of Thrones
% suffer from massive bandwidth load during their premier nights on video
% streaming services. Ideally, creators should be able to write policies that
% allow them to pre-distribute their content while ensuring that it cannot be
% viewed until a pre-set release date.

As an example of a policy with progressive trust, considers
Figure~\ref{fig:time_based_access} which consider content
pre-distribution: a capsule creator writes this policy to
pre-distribute their content while ensuring that the content cannot be
viewed until a pre-set release date. For this use case, we rely on a
trusted remote server for getting the time. Capsule metadata is first
inspected using \texttt{getState()} to check if the content has
already been approved for access by the policy. If this is indeed the
first access to the capsule, using the \texttt{getTime()} API, the
remote server is contacted to get the epoch value and it is compared
to the epoch value in the policy. If the remote epoch stamp is greater
than the time encoded in the policy, the access is approved, and the
metadata is updated using \texttt{setState()} to reflect this. Any
subsequent accesses to the capsule do not involve querying the remote
server for getting the time.

% In situations where the data-owner wishes to exercise greater control over who
% gets to access their data, relying on such statically defined policies might not
% be an appealing solution. Data owners might want to push new policies to affect
% change in capsules that have already been created and circulated.  This use-case
% can be serviced by using the trusted server to host new policy versions. The The
% \texttt{updatePolicy()} API can be used by the trusted application to get an
% updated version from the remote server.

% This mechanism, however, has an inherent delay between the when the owner
% decides to change the policy for a capsule, and the change getting published
% since it is contingent on the owner writing and releasing a new policy
% version. This non-trivial amount of work can lead to the data owners not
% updating the policies in time.

% A mechanism for asking the owner for access in real-time could alleviate this
% effort and provide a finer-grained control on data access. To this end, we
% extended the policy API to introduce a new remote location to seek permission
% from - Twitter. The data owner embeds her Twitter handle in the policy and sets
% the location for metadata operation to "POLICY\_REMOTE\_SERVER". The key for the
% \texttt{getState()} operation is set to "twitter:<username>". The "twitter:"
% string in the key is used to distinguish this as a Twitter request.

% When the remote server receives a \texttt{getState()}, and identifies the key to
% be a twitter handle, it invokes the Twitter Direct Message API to send an open
% request to the data owner. When the data owner receives this response, they can
% reply to "yes" or "no" to approve or decline the access request
% respectively. The server then returns the response to the Trusted Application
% which then returns the appropriate action.

% \textbf{Allowing access to data in agreed-upon locations}: This policy can be
% used by data-owners to control \emph{where} their data can be
% accessed. Enterprises may wish to exercise control on where their employees may
% be able to access their sensitive data.
% {
% \pu{We need to identify why simply
%   blocking access to copying data does not work - This is usually done by using
%   device-level policies that block ways which can be used to copy the data, say,
%   blocking access to USB media.}

\begin{figure*}[t]
\begin{lstlisting}[language={[5.1]Lua} ,basicstyle=\small,showstringspaces=false,numbers=left,stepnumber=1,tabsize=1]
longitude = 1250
latitude = 200
range = 10

function evaluate_policy( op )
    if op == POLICY_OP_OPEN or op == POLICY_OP_CLOSE then
        long, lat, err = getLocation( POLICY_LOCAL_DEVICE )
            if err ~= POLICY_NIL then
                comment = "Failed to getLocation"
                return false
            end
            if math.abs(long - longitude) <= range
            and math.abs(lat - latitude) <= range then
                comment = "GPS coordinates in range"
                return true
            else
                comment = "GPS coordinates are not in range"
                return false
            end
    end
end
\end{lstlisting}
\caption{Simple location based access policy}
\label{fig:location_policy}
\end{figure*}


\textbf{Location based redaction}: The policy as specified is fairly restrictive
and blocks all access to the data contents outside a geographical range. There
can be a scenario where the data owners wish to allow the partial revelation of
the data in locations that are not trusted. To fulfill this requirement, the
policy in Figure \ref{fig:location_policy} can be tweaked to allow redaction of
data that is marked confidential by the creator.

The data creator can encapsulate sensitive information in the data within some
policy defined secrecy tags (for example: \texttt{<secret> , </secret>} can be
used). The \texttt{redact()} API can be used in the policy to look for these
tags and replace the text contained therein with a replacement string. This
policy is shown in Figure \ref{fig:location_redaction_policy}.

\begin{figure*}[t]
\begin{lstlisting}[language={[5.1]Lua} ,basicstyle=\small,showstringspaces=false,numbers=left,stepnumber=1,tabsize=1]
replaceVar = "REDACTED"
longitude = 1250
latitude = 200
range = 10
startTag = "<secret>"
endTag = "</secret>"
function evaluate_policy( op )
    if op == POLICY_OP_OPEN or op == POLICY_OP_CLOSE then
        long, lat, err = getLocation( POLICY_LOCAL_DEVICE )
        if err ~= POLICY_NIL then
            policy_result = POLICY_NOT_ALLOW
            comment = "Failed to getLocation"
            return
        end
        if math.abs(long - longitude) <= range
        and math.abs(lat - latitude) <= range then
            comment = "GPS coordinates in range"
            policy_result = POLICY_ALLOW
            return
        else
            comment = "GPS coordinates are not in range. Redacting data."
            policy_result = POLICY_NOT_ALLOW
        end
        while s ~= nil and e ~= nil do
            s = string.find(data, startTag, s)
            if s ~= nil then
                e = string.find(data, endTag, s+1)
                if e ~= nil then
                    err = redact( s, e, "replaceVar" )
                    if err ~= POLICY_NIL then
                        policy_result = err
                        return
                    end
                end
            end
        end
\end{lstlisting}
\caption{Location based redaction policy}
\label{fig:location_redaction_policy}
\end{figure*}

% \textbf{Pre-distribution of content}: Popular TV shows such as Game of Thrones
% suffer from massive bandwidth load during their premier nights on video
% streaming services. Ideally, creators should be able to write policies that
% allow them to pre-distribute their content while ensuring that it cannot be
% viewed until a pre-set release date.

% This use-case can be expressed easily in our policy API. To facilitate this use
% case, we rely upon a trusted remote server for getting time. Capsule metadata is
% first inspected using \texttt{getState()} to check if the content has already
% been approved for access by the policy. If this is indeed the first access to
% the capsule, using the \texttt{getTime()} API, the remote is contacted to get
% the epoch value and it is compared to the epoch value in the policy. If the
% remote epoch stamp is greater than the time encoded in the policy, the access is
% approved, and the metadata is updated using \texttt{setState()} to reflect
% this. Any subsequent accesses to the capsule do not involve querying the remote
% server for getting the time.

\begin{figure*}[t]
\begin{lstlisting}[language={[5.1]Lua} ,basicstyle=\small,showstringspaces=false,numbers=left,stepnumber=1,tabsize=1]
-- remote server information
remote_server = "198.162.52.127:3490"
-- return keywords
policy_result = POLICY_NOT_ALLOW
comment = ""

-- policy-specific keywords
open_time = 1523338041
opened = "opened"

function evaluate_policy( op )
    if op == POLICY_OP_OPEN then
        r, err = getState( opened, POLICY_CAPSULE_META )
        if r == "true" then
            return true
        else
            curr_time, err = getTime( POLICY_REMOTE_SERVER )
        end
        if err ~= POLICY_NIL then
            policy_result = err
            comment = "Failed to get time from remote server"
            return false
        end
        if curr_time > open_time then
            err = setState( opened, "true", POLICY_CAPSULE_META )
            if err ~= POLICY_NIL then
                policy_result = err
                comment = "Failed to update capsule metadata"
                return false
            end
            return true
        end
    end
end
\end{lstlisting}
\caption{Policy to allow content pre-distribution}
\label{fig:time_based_access}
\end{figure*}

% This policy can also be easily tweaked to revoke access to the file after a pre-determined amount of time has elapsed since the first time the capsule was successfully accessed. In this scenario, the default policy action is \texttt{POLICY\_ALLOW} and the policy transitions to the \texttt{POLICY\_NOT\_ALLOW} once the view window expires. Such a policy can be useful when sharing documents, for instance - college transcripts, that contain sensitive personal identifiers like Social Security Numbers (SSNs).
